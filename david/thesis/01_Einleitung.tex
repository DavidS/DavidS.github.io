\chapter{Einleitung}
\thispagestyle{empty}
\label{Kapitel_Einleitung}

Sp�testens seit der in Ton verewigten Buchhaltung der Sumerer erleichtern sich
Menschen mit Datenaufzeichnungen das Leben. Um immer neuen Anforderungen zu
gen�gen, wurden immer komplexere Verfahren entwickelt. Diese Entwicklung hat
bis zum heutigen Tag mit objekt-orientiertem Design und relationalen
Datenmodellen zwei orthogonale Methoden hervorgebracht.

\begin{itemize}
\item Objekt-orientierte Architekturen stellen die \emph{Beziehungen} von
Objekten in den Vordergrund. Diese \emph{vernetzte Darstellung} erm�glicht eine
realit�tsnahe Modellierung von Sachverhalten, die die Abbildung von
\emph{komplexen Zusammenh�ngen} erm�glicht.

\item Bei relationalen Datenmodellen steht die \emph{Effizienz} der
Datenverwaltung im Mittelpunkt. Die daf�r gew�hlte Einschr�nkung auf
\emph{starre, tabellarische Darstellungen} erm�glicht die Verwaltung
\emph{gro�er Datenmengen}.
\end{itemize}

Um komplexe Zusammenh�nge und Abl�ufe auf gro�en Datenmengen umzusetzen, werden
aber Techniken aus beiden Bereichen ben�tigt. Mittels objekt-relationaler
Abbildungen versucht man die besten Teile aus beiden Welten zu vereinen.
Die dabei verwendeten Techniken und Werkzeuge sind zentrales Thema dieser
Diplomarbeit.

\TODO{Korrektheit �berpr�fen}
Kapitel~\ref{Kapitel_Grundlagen} beschreibt die g�ngigen relationalen und
objekt-orientierten Grundlagen, dazu wird eine Beispieldatenbank vorgestellt.
In Kapitel~\ref{Kapitel_Konzepte} werden die verschiedenen M�glichkeiten der
objekt-relationalen Abbildungen gegen�bergestellt.
Kapitel~\ref{Kapitel_Produkte} stellt die ausgew�hlten Methoden -- reines SQL
und Java -- und die Produkte -- SimpleORM und Hibernate -- vor.
Anschliessend werden diese Methoden und Produkte anhand der Beispieldatenbank
in Kapitel~\ref{Kapitel_Gegenueberstellung} gegen�bergestellt.
Kapitel~\ref{Kapitel_Vergleich} beurteilt die Implementierungen in Hinblick auf
die klassischen Definitionen von objekt-orientierten Datenbanksystemen.
Kapitel~\ref{Kapitel_Zusammenfassung} fasst die gewonnenen
Erkenntnisse und Erfahrungen noch einmal zusammen. Ein kurzer Ausblick auf die
aktuellen Entwicklungen und die Einbettung in gr��ere Projekte schlie�t diese
Arbeit ab.


