\chapter{Zusammenfassung}
\thispagestyle{empty}
\label{Kapitel_Zusammenfassung}

Reines SQL ist immer noch un�bertroffen in Kompaktheit und reiner
Abfrageleistung. Im Gegensatz zur urspr�nglichen Absicht, SQL als
Benutzerschnittstelle anzubieten, ist die Text- und Anweisungs-orientierte
Arbeitsweise nicht mehr Endnutzer-tauglich. Moderne Anwendungen -- sei es auf
Web- oder Desktopbasis -- bauen jedoch auf objekt-orientierten Sprachen und
Bibliotheken auf. Die besprochene Kluft zwischen den leistungsf�higen -- aber
inflexiblen -- relationalen Systemen und den Objektmodellen kann sowohl durch
selbstprogrammierte �bersetzungsfunktionen als auch durch Fremdprodukte
�berbr�ckt werden.

Die vorgestellten Bibliotheken -- SimpleORM und Hibernate -- zeigen dabei ganz
unterschiedliche Ans�tze, wodurch sie in unterschiedlichen Situationen ihre
St�rken ausspielen k�nnen.

\paragraph{SimpleORM} ist eine leistungsstarke Bibliothek, die dem
Programmierer die
grundlegende Kontrolle �ber die Datenbank aus der Hand nehmen. In der
Abw�gung zwischen Funktionsumfang und Benutzbarkeit zielt SimpleORM auf den
erfahrenen JDBC-Programmierer, der sich die mechanische Arbeit des
Datenbankzugriffes erleichten m�chte ohne die M�glichkeiten der
Stapelverarbeitung zu verlieren. Die knappe, pr�zise
Dokumentation~in~\cite{simpleorm} beschreibt in rund 17 Seiten die gesamte
Programmierschnittstelle und die wichtigsten Eckpfeiler der zugrundeliegenden
Designphilosophie.

SimpleORM eignet sich besonders f�r Projekte mit einfachen Datenstrukturen und
hohen Anforderungen an den Datendurchsatz. Komplexere Beziehungen zwischen Daten
-- wie die Hierarchie in der Beispieldatenbank -- m�ssen selbst entworfen
werden, stellen aber aufgrund der hohen Flexibilit�t der Bibliothek keine
besonderen Hindernisse dar.

\paragraph{Hibernate} baut zwar auch auf JDBC auf, um sich zur Datenbank zu verbinden,
bietet aber die gesamte Datenzugriffsfunktionalit�t selbst an. Durch diese
Kapselung kann Hibernate einen wesentlich intensivere Unterst�tzung der
Datenbankkommunikation anbieten, erfordert aber ebenso eine intensivere
Besch�ftigung mit Hibernate selbst. Die Entkoppelung der Datenbank von den
Gesch�ftsobjekten durch die XML-Abbildungsbeschreibung erlaubt auch die
getrennte Weiterentwicklung von Anwendung und Datenbank. Die Hibernate
Referenz~\cite{hibernate} beschreibt auf �ber 200 Seiten die gesamte
Programmierschnittstelle und das XML Schema der Abbildungsbeschreibung.

Hibernate eignet sich besonders f�r Projekte mit komplexen Datenstrukturen oder
nicht-trivialen Datenbanksituationen. Hier wiederum im Speziellen ist der
Zugriff auf Alt-Systeme hervorzuheben, da der Datenzugriff auf beliebige
Schemaformen erfolgen kann. Der Datendurchsatz von Hibernate blieb zwar hinter den
anderen Bibliotheken zur�ck, daf�r gibt es allerdings eine breite Palette an
Optimierungsparametern und Zusatzprodukten zur Leistungssteigerung. Deren
Konfiguration und Einsatz lohnt sich ebenfalls nur f�r Projekte mit
entsprechenden Anforderungen.


SimpleORM als auch Hibernate erf�llen im Zusammenspiel mit der
Implementierungssprache die meisten Anforderungen, die an objekt-orientierte
Datenbanksysteme gestellt werden. Gegen�ber der h�ndischen Implementierung mit
JDBC werden damit respektable Einsparungen im Bereich des Programmieraufwandes
und der Komplexit�t geboten, dem gegen�ber steht eine Investition in das
Erlernen der Bibliothek.

