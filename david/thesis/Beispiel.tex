\chapter{Beispiel}
% nach jedem \section geh�rt diese Zeile eingef�gt, damit keine Nummerierung
% aufscheint:
\thispagestyle{empty}
\label{Kapitel_Garfield}

\section{Garfield}
So hier ein bisschen Textblabla und ein toller Garfield Comic (Siehe
Abbildung~\ref{Garfield}):

\begin{figure}[t]
\begin{center}
\includegraphics[width=450pt,keepaspectratio]{img/garfield}
% 450 ist die optimale Breite eines Bildes, wenn es jedoch h�her als 390pt ist,
% kann kein Text mehr auf derselben Seite angezeigt werden. Dann beschr�nkt man
% am besten die H�he auf 390pt:
%\includegraphics[height=390pt,keepaspectratio]{img/garfield}
\end{center}
\caption{If it ain't broke, don't fix it!\cite{garfield}}
\label{Garfield}
\end{figure}

\section{Sonstiges}
\subsection{Zitate}
Mit dem Befehl myquotecite kann man auch noch toll Zitate einf�gen:

\begin{myquotecite}{garfield}
Ich bin nicht �bergewichtig, ich bin nur untergro�!
\end{myquotecite}

F�r die Bibtex gibt es noch eine recht gute Webseite, die die Bibtex
Formatierung auflistet.\cite{bibtex}

\subsection{Makefile}
Die Befehle f�rs beigelegte Makefile sind in Tabelle~\ref{makefile} aufgelistet:

\renewcommand{\baselinestretch}{1}
\begin{table}[h]
\begin{tabular}{p{2cm}p{13cm}}
Befehle & Was tun sie? \\
\hline
make & ruft mehrmals "`pdflatex Diplomarbeit"' auf, damit alle Referenzen
stimmen und bindet auch die Bibliographie mit ein (bibtex Diplomarbet) \\
make fast & schnelle Version, wenn man nur Kleinigkeiten ge�ndert hat --
Referenzen und Bibliographie werden nicht eingebunden! \\
make view & �ffnet das pdf file im Acrobat Reader (acroread Diplomarbeit.pdf) \\
make clean & l�scht alle unn�tige Dateien (Table of content, pdfs...). Nur die
tex-files, Bibliographie und Bilder bleiben bestehen. \\
\hline
\end{tabular}
\caption{Wie verwende ich das Makefile?}
\label{makefile}
\end{table}
\renewcommand{\baselinestretch}{1.5}

So, viel Spa�!
