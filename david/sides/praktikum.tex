\documentclass[12pt,a4paper]{article}
\usepackage{german,a4,url}

\title{Praktikum\\"`LVA-spezifischer Webspace f"ur Vortragende"'}
\author{David Schmitt\thanks{e9725491@student.tuwien.ac.at}}

\nonfrenchspacing

\begin{document}

\maketitle

\vfill

\begin{abstract}

Das Ziel dieses Praktikums ist es einen "Uberblick "uber die M"oglichkeiten
f"ur eine Einbettung "`LVA-spezifischen Webspaces f"ur Vortragende"' in die
vorhandene SIDES Umgebung. Im Kapitel \textsl{Anforderungen} werden die
Punkte Authentifikation und Autorisation, Sessionabwicklung,
Life-Cyclemanagement einer LVA, Referenz-Management beim Upload,
statistische Zugriffsauswertungen f"ur Vortragende, Security sowie die
notwendigen technischen Grundlagen diskutiert, die f"ur das Projekt
zentrale Punkte sind.

% Dann werden im Kapitel \textsl{Vorschl"age} konkrete Vorschl"age zur
% Verwendung von Software gemacht, die die geforderte Funktionalit"at
% unterst"utzt oder entsprechend angepa"st werden kann.

%% BZW: ...entsprechende Implementierung...

\end{abstract}

\newpage

\tableofcontents

\newpage

\section{Einleitung}

\section{Anforderungen}
\subsection{Authentifikation und Autorisation}

Wie auf \url{http://www.lzk.ac.at/sides/sides-auth.html} beschrieben sind
die Authentifikation\footnote{Festellung der Person} und die
Autorisation\footnote{Feststellung der Berechtigungen einer Person}
getrennte Abl"aufe.

\subsubsection{Authentifikation}

Zur Authentifizierung kann der sogenannte Auth-Daemon, der via LDAP auf die
X.500 Whitepages\footnote{Siehe \url{http://wp.tuwien.ac.at/}} der TU-Wien
zugreift, genutzt werden.

%% FIXME: Keine Dokumentation f"ur authd vorhanden.

\subsubsection{Autorisation}

Autorisiert soll "uber die Zuordnung von LVAs zu Vortragenden in der SIDES
Datenbank werden. Hierbei ist sowohl der Zugang via Embedded SQL von Oracle
oder via DBD::Oracle\footnote{Siehe
\url{http://search.cpan.org/doc/TIMB/DBD-Oracle-1.06/Oracle.pm}} m"oglich.
Eine Erweiterung des Auth-Daemons w"are ebenfalls eine denkbare Alternative
um die ben"otigte Information "uber einen Socket zu Verf"ugung zu stellen.

\subsubsection{Integration in die Infrastruktur}

\paragraph{DRAFT:}{Die Methode des Filezugriffs muss "uber die gew"ahlten
Methoden authentifizieren und autorisieren k"onnen.}

\subsection{Sessionabwicklung}

Im Rahmen einer Session eines Nutzers sollten Files zu LVAspaces
hinzugef"ugt, "uberschrieben bzw. ge"andert, verschoben, umbenannt und
gel"oscht werden k"onnen. "Ahnliches mu"s f"ur Verzeichnisse m"oglich sein. 
Dabei ist besonders darauf zu achten, da"s auch unge"ubte Nutzer das System
leicht bedienen k"onnen.

% FIXME: Um eine gute Integration zu erreichen m"ussen div. HTML-Editoren auf ihre publishing features hin untersucht werden.

\subsubsection{FTP}

\paragraph{Vorteile: } 
\begin{itemize}
\item{Wird von vielen HTML-Editoren zum Publishen benutzt}
\item{Stabiles Protokoll}
\item{Codebasis vorhanden}
\item{Plattformunabh"angig}
\end{itemize}
\paragraph{Nachteile: }
\begin{itemize}
\item{Nicht trivial in der Anwendung}
\end{itemize}


\begin{description}
\item[proFTPd:]{A powerful replacement for wu-ftpd, this File Transfer
Protocol daemon supports hidden directories, virtual hosts, and
per-directory "`.ftpaccess"' files.
\begin{description}
\item[Homepage:]{\url{www.proftpd.net}}
\item[Implementierung:]{C}
\item[Modular:]{Ja}
\item[Virtuelles Filesystem:]{M"oglich}
\item[Quota:]{M"oglich}
\item[Externe Authentifikation:]{M"oglich}
\end{description}}

\item[pyftpd:]{An advanced ftp server written in python.
\begin{description}
\item[Homepage:]{\url{http://melkor.dnp.fmph.uniba.sk/~garabik/pyftpd.html}}
\item[Implementierung:]{Python}
\item[Modular:]{Ja}
\item[Virtuelles Filesystem:]{M"oglich}
\item[Quota:]{M"oglich}
\item[Externe Authentifikation:]{M"oglich}
\end{description}}

\item[Net::FTPServer:]{Net::FTPServer is a secure, extensible and
configurable FTP server written in Perl.
\begin{description}
\item[Homepage:]{\url{http://search.cpan.org/doc/RWMJ/Net-FTPServer-1.0.7/lib/Net/FTPServer.pm}}
\item[Implementierung:]{Perl}
\item[Modular:]{Ja}
\item[Virtuelles Filesystem:]{Vorhanden, mu"s adaptiert werden}
\item[Quota:]{M"oglich}
\item[Externe Authentifikation:]{M"oglich}
\end{description}}

\end{description}

\subsubsection{SMB}

\paragraph{Vorteile: }
\begin{itemize}
\item{Triviale Anwendung f"ur Windows User}
\item{Codebasis vorhanden}
\end{itemize}
\paragraph{Nachteile: }
\begin{itemize}
\item{Instabiles Protokoll}
\item{Plattformabh"angig}
\item{Probleme unter Win95 mit Username/Passwort}
\end{itemize}

\begin{description}
\item[Samba]
\begin{description}
\item[Homepage:]{\url{http://de.samba.org/}}
\item[Implementierung:]{C}
\item[Modular:]{Nein}
\item[Virtuelles Filesystem:]{via Shares}
\item[Quota:]{Nein}
\item[Externe Authentifikation:]{Nein}
\end{description}
\end{description}

\subsubsection{WebFormular}

\paragraph{Vorteile: }
\begin{itemize}
\item{Einfache Anwendung}
\item{Plattformunabh"angig}
\end{itemize}
\paragraph{Nachteile: }
\begin{itemize}
\item{kaum Codebasis vorhanden}
\item{M"uhsam mehrere Files upzuloaden}
\end{itemize}

\paragraph{Beispiele:} Siehe \url{http://www.rainerhansen.de/webspace.htm}

\subsubsection{WebDAV}

\paragraph{Vorteile: }
\begin{itemize}
\item{Einfache Anwendung}
\end{itemize}
\paragraph{Nachteile: }
\begin{itemize}
\item{keine Codebasis vorhanden}
\item{Applikationsabh"angig}
\end{itemize}


\subsection{Life-Cyclemanagement}

\paragraph{TODO}{Es m"ussen "Uberlegungen zur Erstellung, Benutzung,
Rechtevergabe, Quota"anderungen, Semesterwechsel und L"oschung der LVA
angestellt werden. Diese Vorg"ange m"ussen in die bestehenden Automatismen
eingebunden werden und entsprechend automatisiert durchf"uhrbar sein.  Wie
die restlichen Abl"aufe im SIDES sollten auch zu diesen Abl"aufen ohne
h"andische Eingriffe Verst"andigungen an die Betroffenen verschickt
werden.}

\subsection{Referenzmanagement}

Um die Benutzung des Systems m"oglichst einfach und transparent zu
gestalten sollte das System daf"ur sorgen, dass Referenzen zwischen
Dokumenten nach M"oglichkeit erhalten bleiben. Insbesondere sollte
versucht werden Referenzen in das Filesystem des Nutzers auf korrekte
Referenzen in den LVAspaces umzuschreiben. Komplement"ar dazu sollte mit
einem Linkchecker die Seiten "uberpr"uft werden um Fehler, die durch
einfaches Umschreiben nicht behoben werden konnten abzufangen und den
Benutzer m"oglichst proaktiv darin zu Unterst"utzen, den Fehler zu
beheben.

\subsubsection{Referenzen umschreiben}

\paragraph{TODO}{Einbindung in Fileupload?}

\begin{itemize}
\item{Liste aller ung"ultigen Referenzen extrahieren (\url{file://}, \dots)}
\item{Ersetzen von "`$\backslash$"' (Backslash) durch "`/"' (Schr"agstrich)}
\item{Liste von Dateien im Verzeichnis der LVA mit "`ung"ultigen"'
Referenzen vergleichen und Treffer ausbessern}
\item{Restliche Referenzen als defekt markieren?}
\end{itemize}

\subsubsection{Referenzchecking}

\paragraph{DRAFT:}{Da zu erwarten ist, dass durch das Umschreiben von
Referenzen nicht alle Fehler behoben werden k"onnen und auch im Laufe der
Zeit externe Links ung"ultig werden, ist es angebracht eine zweite
Verifikationsphase anzuschliessen, die mit einem
Standardtool\footnote{urlbot, linkbot, o."a.} zu "uberpr"ufen und die
Resultate den Vortragenden zur Verf"ugung zu stellen. Im Sinne einer
Verbesserung\footnote{haha.} der Qualit"at des HTML-Angebotes k"onnte man
an dieser Stelle auch einen HTML-Checker wie
\url{http://valse.tuwien.ac.at/} einbinden.}

\subsection{Zugriffsstatistiken}

Ein Tool wie analog oder webalizer soll benutzt werden um nach LVA
aufgeschl"usselte Zugriffstatistiken zu erzeugen.

\subsection{Security}

K"onnen durch den Einsatz des Systems Resourcen mi"sbraucht werden?
Welche besonderen Vorkehrungen zur Absicherung des Systems sind
notwendig?

\subsection{Technische Infrastruktur}

\subsubsection{Vorhandene}

Liste der aktuell eingesetzten Technologien und Produkte:
\begin{description}
\item[Webserver:]{Apache\footnote{Siehe \url{http://www.apache.org/}}}
\item[CGI-Umgebung:]{mod\_perl\footnote{Siehe \url{http://perl.apache.org/}}}
\item[Datenbank:]{Oracle 8\footnote{Siehe \url{http://www.oracle.com/}}}
\item[DB-Access:]{oraperl/DBD::Oracle\footnote{Siehe 
\url{http://search.cpan.org/doc/TIMB/DBD-Oracle-1.06/Oracle.pm}}}
\item[Securityframework:]{SIDES\footnote{Siehe \url{http://www.lzk.ac.at/sides/sides-auth.html}}/Whitepages/authd}
\end{description}

\subsubsection{Ben"otigte Ressourcen}

Sch"atzungsweise 1500 LVAs der insgesamt 5000 an der TU-Wien angebotenen
LVAs werden das Service in Anspruch nehmen.

Daraus ergeben sich folgende Sch"atzungen f"ur den Platzbedarf der aktiven
LVAspaces:
\begin{description}
\item[Fixbedarf Statistiken:]{$500kB*1500 = 750MB$}
\item[1MB pro aktiver LVA:]{$1MB*1500 = 1.5GB$}
\item[10MB pro aktiver LVA:]{$10MB*1500 = 15GB$}
\item[Maximalbedarf der aktiven LVA]{$20MB*1500 = 30GB$}
\end{description}

Es kann also mit einem Bedarf von rund 1GB sofort und bis zu maximal 15GB
Platzbedarf nach der Anlaufphase gerechnet werden. Das ist nat"urlich stark
von der tats"achlichen Nutzung abh"angig. Einen Bedarf von 3GB nach dem
ersten Semester halte ich f"ur realistischer, kann diese Zahlen jedoch
nicht mit konkreten Hinweisen untermauern. Sollte eher der maximale Rahmen
ausgesch"opft werden so ist mit hoher Wahrscheinlichkeit sowieso die
Anschaffung neuer Hardware erforderlich, da nach meiner Information
momentan keine 15GB zur Verf"ugung stehen. Nach dem heutigen Stand der
Technik sollte eine solche Erweiterung jedoch keine technischen Probleme
verursachen.

\section{Vorschl"age}

\subsection{Software}

\subsection{Implementierung}

\section{Zusammenfassung}

\end{document}
