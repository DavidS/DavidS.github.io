\documentclass[12pt,a4paper]{article}

\usepackage{german,a4,url}


\title{Besprechungsprotokoll\\Freitag, 2001-04-06}
\author{David Schmitt\thanks{e9725491@student.tuwien.ac.at}}

\nonfrenchspacing

\begin{document}

\maketitle

\vfill

\begin{abstract}

Lagebesprechung zur Kl"arung des Umfangs und Inhaltes des Praktikums
"`LVA-spezifischer Webspace f"ur Vortragende"'. Es wurden grundlegende
Parameter und folgende Inhalte festgelegt: Authentifikation und
Autorisation, Ressourcen, Sessionabwicklung, Life-Cyclemanagement einer
LVA, Referenz-Management beim Upload, statistische Zugriffsauswertungen
f"ur Vortragende, Security sowie die notwendigen technischen Grundlagen.
Au"serdem wurde ein vorl"aufiger Zeitplan f"ur die Abwicklung des
Praktikums skizziert und ein "Uberblick "uber die vorhandene technischen
Infrastruktur gegeben.

\end{abstract}

\newpage

\section{Umfang und Form}

Das Praktikum "`LVA-spezifischer Webspace f"ur Vortragende"' sollte in
einem Design- und Technikvorschlag zur Umsetzung des Projektes
"`LVA-spezifischer Webspace f"ur Vortragende"' resultieren. Ausgehend von
den konkreten Vorschl"agen und Ideen des Papers sollte eine rasche und
geradlinige Durchf"uhrung des Projektes m"oglich sein. Das Paper soll in
\LaTeX\ abgefasst sein.

\section{Authentifikation und Autorisation}

Wie auf \url{http://www.lzk.ac.at/sides/sides-auth.html} beschrieben sind
die Authentifikation\footnote{Festellung der Person} und die
Autorisation\footnote{Feststellung der Berechtigungen einer Person}
getrennte Abl"aufe. Zur Authentifizierung soll der sogenannte Auth-Daemon,
der via LDAP auf die X.500 Whitepages\footnote{siehe
\url{http://wp.tuwien.ac.at/}} der TU-Wien zugreift, genutzt werden.
Autorisiert soll "uber die Zuordnung von LVAs zu Vortragenden in der SIDES
Datenbank werden.

\section{Ressourcen}

Zur Zeit gibt es an der TU-Wien ca.~5000 LVAs. Jeder LVA sollten 20MB
Speicherplatz zur Verf"ugung gestellt werden. Es wird jedoch erwartet, dass
nur 1500 LVAs das Service in Anspruch nehmen. Dazu kommen noch ca 250-500kB
f"ur Zugriffsstatistiken bei genutzten LVAs und das Indextemplate f"ur jede
LVA.

\section{Sessionabwicklung}

Wie soll der Zugriff auf die LVAspaces aussehen? Vorgeschlagen wurde ftp,
WebDAV, smb oder Webupload. Es ist dabei besonders darauf zu achten, da"s
das System vorallem von computerfremden Vortragenden genutzt werden soll
und daher das Interface so einfach wie m"oglich gestaltet werden sollte.
Auch ist darauf zu achten, da"s nur die LVAspaces sichtbar sein sollten,
auf die der angemeldete Benutzer Zugriff hat. Der Benutzer mu"s alle
"ublichen Dateioperationen durchf"uhren k"onnen: schreiben, lesen,
umbenennen, l"oschen, Verzeichnisse anlegen, umbenennen und l"oschen.

\section{Life-Cyclemanagement}

Es m"ussen "Uberlegungen zur Erstellung, Benutzung, Rechtevergabe,
Quota"anderungen, Semesterwechsel und L"oschung der LVA angestellt werden.
Diese Vorg"ange m"ussen in die bestehenden Automatismen eingebunden werden
und entsprechend automatisiert durchf"uhrbar sein.  Wie die restlichen
Abl"aufe im SIDES sollten auch zu diesen Abl"aufen automatisiert
Verst"andigungen an die Betroffenen verschickt werden.

\section{Referenzmanagement}

Um die Benutzung des Systems m"oglichst einfach und transparent zu
gestalten sollte das System daf"ur sorgen, dass Referenzen zwischen
Dokumenten nach M"oglichkeit erhalten bleiben. Insbesondere sollte versucht
werden Referenzen in das Filesystem des Nutzers auf korrekte Referenzen in
den LVAspaces umzuschreiben. Komplement"ar dazu sollte mit einem
Linkchecker die Seiten "uberpr"uft werden um Fehler, die durch einfaches
Umschreiben nicht behoben werden konnten abzufangen und den Benutzer
m"oglichst proaktiv darin zu Unterst"utzen, den Fehler zu beheben.

\section{Zugriffsstatistiken}

Ein Tool wie analog oder webalizer soll benutzt werden um nach LVA
aufgeschl"usselte Zugriffstatistiken zu erzeugen.

\section{Security}

K"onnen durch den Einsatz des Systems Resourcen mi"sbraucht werden?  Welche
besonderen Vorkehrungen zur Absicherung des Systems sind notwendig?

\section{Technische Infrastruktur}

\subsection{Vorhandene}

Liste der aktuell eingesetzten Technologien und Produkte:
\begin{description}
\item[Webserver:]{Apache\footnote{siehe \url{http://www.apache.org/}}}
\item[CGI-Umgebung:]{mod\_perl\footnote{siehe \url{http://perl.apache.org/}}}
\item[Datenbank:]{Oracle 8\footnote{siehe \url{http://www.oracle.com/}}}
\item[DB-Access:]{oraperl/DBD::Oracle\footnote{siehe 
\url{http://gd.tuwien.ac.at/languages/perl/CPAN/modules/by-category/07_Database_Interfaces/Oraperl/DBD-Oracle-1.06.readme}}}
\item[Securityframework\footnote{siehe \url{http://www.lzk.ac.at/sides/sides-auth.html}}:]{SIDES/Whitepages/authd}
\end{description}

\section{Zeitplan}

Enzi wird am Dienstag, 2001-04-17 aus dem Urlaub zur"uck sein. Dann sollen
dir bis dato gesammelten Informationen vor dem SIDES Team pr"asentiert und
diskutiert werden um sie in der zweiten Woche nach Ostern weiter zu
verfeinern und in eine akzeptable Form zu bringen.

\section{Zusammenfassung}

Nocheinmal kurz zusammengefa"sst die wichtigsten zu behandelnden Punkte des
Praktikums:
\begin{itemize}
\item{Zusammenfassung "uber M"oglichkeiten des Projektes 
	"`LVA-spezifischer Webspace f"ur Vortragende"'}
\item{Authentifikation via authd/LDAP}
\item{Autorisation aus SIDES-Datenbank}
\item{Darstellung/Abwicklung einer Session}
\item{Life-Cyclemanagement: Anlegen, l"oschen, editieren, Semesterende,
	LVA-Ende}
\item{Referenzenmanagement: innere Struktur von Dokumentsammlungen
	erhalten}
\item{Zugriffstatistiken}
\item{Technische Infrastruktur: ftpd, Quotas, Ressourcen}
\item{Erste Pr"asentation am Dienstag, 2001-04-17}
\end{itemize}

\end{document}
